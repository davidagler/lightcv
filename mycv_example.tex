\documentclass[p1noheader, 11pt, darkmode]{lightcv}
\usepackage{ebgaramond-maths}
\hypersetup{linkcolor=magenta!30, urlcolor=magenta}

\author{David W. Agler}
\begin{document}

\ContactInfo[2][
\crow [\faEnvelopeO] dwa132@psu.edu
 \crow[\faGlobe]  \href{www.davidagler.com}{davidagler.com}
 \crow[{\faYoutube}] \href{https://www.youtube.com/davidagler}{youtube.com/davidagler}
 \crow[\faGithub] \href{https://www.github.com/davidagler}{github.com/davidagler}
 \crow[\faArchive] \today
][
\crow[\faInstitution] The Pennsylvania State University
\crow Philosophy Department
\crow 227B Sparks Building
\crow University Park, PA 16802
\crow[\faGlobe] \href{https://philosophy.la.psu.edu/people/dwa132/}{Faculty Profile}
]

%\faInstagram % \faTwitter
\begin{dated}[Positions]
    2024-- & \textit{Associate Teaching Professor of Philosophy}, Penn State. \\
    2014-- & \textit{Teaching \& Learning with Technology Coordinator}, Penn State. \\
    2012--2024 & \textit{Assistant Teaching Professor of Philosophy}, Penn State. \\
    2007--2007 & \textit{Lecturer in Philosophy}, Indiana University--Purdue University Indianapolis, IUPUI.\\
\end{dated}
\begin{dated}[Education]
    2007--2012 & \textit{Ph.D. in Philosophy}, Penn State, University Park, PA.\\
    & Dissertation: \textit{\href{https://etda.libraries.psu.edu/files/final_submissions/7042}{Pragmatic Minimalism: A Defense of Formal Semantics}}.\\
    & Committee: Vincent Colapietro (co-chair), Emily Grosholz (co-chair), Christopher Long, and Linda Selzer.\\
    2004--2010 & \textit{M.A. in American Philosophy}, IUPUI, Indianapolis, IN.\\
    & Thesis title: \textit{\href{https://scholarworks.iupui.edu/server/api/core/bitstreams/2e288035-b7ed-49b8-8875-1821b75a402e/content}{Vagueness and Its Boundaries: A Peircean Theory of Vagueness}}.\\
    & Committee: Cornelis de Waal (chair), André De Tienne, and Nathan Houser.\\
    2001--2004 & \textit{B.A. in Philosophy}, IUPUI, Indianapolis, IN.\\
    2001--2004 & \textit{B.A. in English}, IUPUI, Indianapolis, IN.
\end{dated}

\section*{Research and Teaching Interests}

Classical American Philosophy (esp. Peirce, Royce), Philosophy of Language, Symbolic Logic, Metaphysics (esp. temporal ontology), Critiques of Wellness.

\begin{rlist}[Books]
    \item \textit{In progress.} Agler, D. W. (c. 2024). \textit{Symbolic Logic: Syntax, semantics, proof.} \textbf{2nd ed}. Lanham: Rowman \& Littlefield.
    \item Agler, D. W. (2013). \textit{\href{https://rowman.com/ISBN/9781442217416}{Symbolic Logic: Syntax, semantics, proof.}} Lanham: Rowman \& Littlefield.
\end{rlist}

\begin{rlist}[Articles]
    \item Agler, D. W. (2023). \href{http://polanyisociety.org/TAD%20WEB%20ARCHIVE/TAD49-2/Agler-49-2.pdf}{Reflections on Guide to Personal Knowledge.} \textit{Tradition and Discovery} 49(2), 11-17.
    \item Stango, M. \& Agler, D. W. (2017). \href{https://revistas.uva.es/index.php/sociotecno/article/view/1707}{Human body, enhancement, and the missing technomoral virtue.} \textit{Sociología y tecnociencia}, 8(1), 43–59.
    \item Pollock, R. \& Agler, D. W. (2016). \href{http://onlinelibrary.wiley.com/doi/10.1111/papq.12118/abstract}{Hume and Peirce on the ultimate stability of belief.} \textit{The Pacific Philosophical Quarterly}, 97(2), 245–269.
    \item Agler, D. W. \& Stango, M. (2015). \href{http://dx.doi.org/10.1017/hgl.2015.15}{W. T. Harris, Peirce, and the charge of nominalism.} \textit{Hegel Bulletin}, 36(2), 135–158.
    \item Agler, D. W. (2014). \href{https://philpapers.org/rec/AGLEFW}{Emergence from within and without: Juarrero on Polanyi’s account of the external origin of emergence.} \textit{Tradition and Discovery: The Polanyi Society Periodical}, 40(3), 23–35.
    \item Agler, D. W. (2013). \href{https://doi.org/10.1515/sem-2013-0012}{Peirce and the specification of borderline vagueness.} \textit{Semiotica}, 193, 195–215.
    \item Agler, D. W. (2013). \href{https://philpapers.org/rec/AGLWEC}{What engineers can do but physicists can’t: Polanyi and Margitay on machines.} \textit{Tradition and Discovery: The Polanyi Society Periodical}, 39(2), 22–26.
    \item Agler, D. W. \& Durmuş, D. (2013). \href{https://muse.jhu.edu/article/536356/}{Christine Ladd-Franklin: Pragmatist feminist.} \textit{Transactions of the Charles S. Peirce Society: A Quarterly Journal in American Philosophy}, 49(3), 299–321.
    \item Agler, D. W. (2012). \href{https://philpapers.org/rec/AGLPAP}{Polanyi and Peirce on the Critical Method.} \textit{Tradition and Discovery: The Polanyi Society Periodical}, 38(3), 13–30.
    \item Agler, D. W. (2011). \href{https://muse.jhu.edu/article/419995}{Peirce’s direct, non-reductive contextual theory of names.} \textit{Transactions of the Charles S. Peirce Society: A Quarterly Journal in American Philosophy}, 46(4), 611–640.
    \item Agler, D. W. (2010). \href{https://doi.org/10.1177/0270467610361230}{The UFAIL approach: unconventional weapons and their ‘unintended’ medical effects.} \textit{Bulletin of Science, Technology \& Society}, 30(2), 103–112.
    \item Agler, D. W. (2006). \href{https://www.pdcnet.org/pdc/bvdb.nsf/purchase?openform&fp=cpsem&id=cpsem_2006_0101_0112}{The role of replication in the growth of symbols.} In T. J. Prewitt \& B. Smith (Eds.), \textit{Semiotics 2006}, 101–112.
    \item Agler, D. W. (2006). The symbolic self (o eu simbólico). \textit{Cognitio-Estudos: Revista Electrônica de Filosofia}, 3(1), 1–9.
\end{rlist}

\begin{rlist}[Reviews and Other Writings]
\item Agler, D. W. (2019). \href{https://muse.jhu.edu/pub/3/article/751300}{Review of Pragmatism and Vagueness: The Venetian Lectures; Edited by Giovanni Tuzet by Claudine Tiercelin}. \textit{Transactions of the Charles S. Peirce Society: A Quarterly Journal in American Philosophy} 55(4), 458-463.
\item Agler, D. W. (2013). \href{https://www.oxfordreference.com/display/10.1093/acref/9780199797745.001.0001/acref-9780199797745-e-0204?rskey=R3wQuA&result=7}{Hitchcock, Ethan Allen}. In J. R. Shook \& C. de Waal (Eds.), \textit{The Dictionary of Early American Philosophers}. 500 words. Bristol: Thoemmes Continuum.
\item Agler, D. W. (2013). \href{https://philosophynow.org/issues/94/What_Are_The_Most_Important_Things_To_Know}{What are the Most Important Things to Know?} \textit{Philosophy Now}, 94.
\item Agler, D. W. (2012). \href{https://philpapers.org/rec/AGLPOP-2}{Review of Robert B. Brandom, Perspectives on Pragmatism: Classical, Recent, and Contemporary.} \textit{Tradition and Discovery: The Polanyi Society Periodical}, 38(3), 69–71.
\item Agler, D. W. (2011). \href{https://philpapers.org/rec/AGLBMJ-2}{Review of Alice Crary, Beyond Moral Judgment.} \textit{The Pluralist}, 6(2), 103–110.
\item Agler, D. W. (2011). \href{https://doi.org/10.5840/traddisc2011/201238115}{Review of Laura E. Weed, The Structure of Thinking: A Process-oriented Account of Mind.} \textit{Tradition and Discovery: The Polanyi Society Periodical}, 38(1), 66–69.
\item Agler, D. W. (2010). Book notice for Writings of Charles S. Peirce: A Chronological Edition (1890–1892). \textit{The Reasoner}, 4(6), 93.
\end{rlist}

\begin{rlist}[Talks]
    \item Agler, D. W. (2021). Disjunction and Biconditional Derivation Rules in Propositional Logic. John Carroll University. 11 Nov 2021.
    \item Agler, D. W. (2021). Disjunction and Biconditional Derivation Rules in Propositional Logic. John Carroll University. 13 April 2021.
    \item Agler, D. W. (2020). Derived Rules in Propositional Logic Proofs. John Carroll University. 17 Nov 2020.
    \item Agler, D. W. (2018). Reductionism in Mathematics and Psychology. Penn State. 26 Nov 2018.
    \item Agler, D. W. (2018). Logic and the Method of Science: Hypothetico-Deductivism, Falsification, and the Duhem-Quine Problem. Penn State. 10 Oct 2018.
    \item Agler, D. W. (2014). Two criticisms of the Cartesian maxim: Peirce’s rejection of the method of doubt. The Society for the Advancement of American Philosophy. The Richard Stockton College of New Jersey. Galloway Township, NJ.
    \item Agler, D. W. (2014). Vagueness, language, and the problem of heaps: Peirce’s dissolution of the Sorites paradox. The Society for the Advancement of American Philosophy. Denver, CO.
    \item Agler, D. W. (2012a). Peirce’s critiques of the Cartesian maxim. The West Virginia Philosophical Society. Westminster College. New Wilmington, PA.
    \item Agler, D. W. (2011). Comments on Christopher Gibilisco’s “David Lewis and Contingent Second-order Predication”. 10 sept. 2011. Pittsburgh Area Philosophy Colloquium. Washington and Jefferson
    College
    \item Agler, D. W. (2011). Modularity and minimalism. Pittsburgh Area Philosophy Colloquium. Washington and Jefferson College. Washington, PA.
    \item Agler, D. W. (2010). Peirce and Polanyi on doubt. Polanyi Society Annual Meeting. Atlanta, GA.
    \item Agler, D. W. (2006a). Comments on Albert Spencer’s “Am I My Brother’s Keeper? Royce and Dewey on the Community’s Responsibility for the Lost Individual. The 6th Annual Donald G. Wester Conference: Josiah Royce on Ethics and Community. Oklahoma City, OK. 8 April, 2006.
    \item Agler, D. W. (2006). The role of replication in the growth of symbols. 31st Annual Meeting of the Semiotic Society of America. Purdue University. West Lafayette, IN.
    \item Agler, D. W. (2005). The symbolic self. 8th International Meeting on Pragmatism. Pontifical Catholic University of Sao Paulo, Brazil.
    \item Agler, D. W. (2005b). The symbolic self. The Institute for American Thought. Indianapolis, IN.
\end{rlist}

\begin{rlist}[Programs]
    \item Agler, D. W. (2024). \href{https://github.com/davidagler/truthtablegenerator}{Propositional Logic Syntax Checker and Truth Table Generator} (Python).
    \item \textit{In Progress.} Agler, D. W. (2024). CanvasGrader: A Tkinter GUI for Grading CANVAS Essay Questions. Github.
    \item Agler, D. W. (2023). \href{https://github.com/davidagler/lightcv}{LightCV: A \LaTeX\ Class for Curriculum Vitae}. Github. (This CV is made with LightCV).
    \item Agler, D. W. (2023). \href{https://github.com/davidagler/proofpack}{Proofpack: A \LaTeX\ Package for Typesetting Natural Deduction Proofs}. Github.
\end{rlist} 

\begin{dated}[Courses Taught]
    PHIL 001&       Introduction to Philosophy\\
    PHIL 003&   	 Philosophy of Well-being\\
    PHIL 004&     The Human Condition\\
    PHIL 010&   	 Critical Thinking\\
    PHIL 012&   	 Symbolic Logic\\
    PHIL 013&   	 Philosophy of Nature and the Environment\\
    PHIL 083&   	 Bioethics (First-Year Seminar)\\
    PHIL 101&   	 American Philosophy\\
    PHIL 102&   	 Existentialism and European Philosophy\\
    PHIL 103&   	 Ethics\\
    PHIL 107&   	 Philosophy of Technology\\
    PHIL 110&   	 Philosophy of Science\\
    PHIL 123&   	 Introduction to Media Ethics (First-Year Seminar)\\
    PHIL 125W&   	 Epistemology (Writing Intensive)\\
    PHIL 126&   	 Metaphysics\\
    PHIL137N&     Introduction to Philosophy Through Health and Sport.\\
    PHIL 205&   American Philosophy\\
    PHIL 297&   	 Philosophy of Sport\\
    PHIL 401&   	 American Philosophy\\
    PHIL 426W&   	 Metaphysics (Writing Intensive)\\
    PHIL 512&   	 Graduate Logic\\
\end{dated}

\begin{rlist}[Undergraduate Independent Studies] 
\item Topics in Symbolic Logic -- Alex Grigas, John Ouligian, Gretha Dos Santos, Madison Phillips
\item Vagueness and the Line-Drawing Fallacy -- Matthew Thompson
\item Logics of Vagueness -- Seongtaeg Kang
\item Logic and Logic Games -- Chloe DeOnna
\item Modal Logic -- Michael Challis
\item Classical American Pragmatism and Implicit Bias -- Nicholas Charles
\item Decision theory and the holdout problem -- Megan Nuggihalli
\item Computation and Complexity -- David Friedman and Andres De La Fuente
\item Philosophy of Time -- Emily Dorshaw and John Michael Gurklis
\end{rlist}

\begin{rlist}[Courses Designed]    
    \item PHIL010 - Critical Thinking (web). Over 600 pages of instructional content, developed online exams, practice quizzes, quizzes, discussion forum, instructor’s manual, approximately
    26 lightboard videos, and supplemental video tutorials.
    \item PHIL012 - Symbolic Logic (web). With Mark Fisher. Wrote content, developed online exams, quizzes, instructor’s manual, approximately 50 logic videos, and technical video tutorials.
    \item PHIL137N – \href{https://bulletins.psu.edu/university-course-descriptions/undergraduate/phil/}{Introduction to Philosophy Through Health and Sport}. Inter-domain course that introduces students to philosophical ideas and contextualizes them in relation to the concepts of health, wellness, and sport.
\end{rlist}

\begin{catsec}[Professional Service, Awards, Development, Certificates]
\catlist{Certificates}{
    \item \textit{In Progress.} Harvard's CS50P: Introduction to Programming with Python. \href{https://cs50.harvard.edu/certificates/2fffce4c-098d-4acc-89eb-d67934eb3169}{Certificate}
    \item Online Course Authoring Certificate. Penn State World Campus Online Faculty Development. Completed: 2023
    \item Foundations for Online Teaching Certificate. Penn State World Campus Online Faculty Development. Awarded: 2022
    \item Instructional Practice Certificate. Penn State World Campus Online Faculty Development. Completed: 2020}
\catlist{Awards}{
    \item PSU Philosophy Department Joseph J. Kockelmans Award in Philosophy. Awarded 2012.
    \item Harold F. Martin Graduate Assistant Outstanding Teaching Award – Penn State Graduate School and Office of the Vice President and Dean for Undergraduate Education. Awarded: 2011.
    \item Jean Martin Maxwell Prize for best M.A. Thesis containing a contribution to American Philosophy. Awarded: 2010.
    \item Graduate Fellowship. The Pennsylvania State University. Awarded: 2007-2012.
    \item IUPUI Summer Research Fellowship. The Peirce Edition Project. Awarded: 2006.
    \item IUPUI Research \& Graduate Fellowship. IUPUI Philosophy Department \& Peirce Edition Project. Awarded: 2005-2006.
    \item Co-winner of the IUPUI Philosophy Department Essay Competition, “The Symbolic Self”.}
\catlist{Referee}{
    \item Semiotica
    \item Routledge
    \item Journal of Speculative Philosophy
    \item Teaching Philosophy
    \item Rowman \& Littlefield Publishing Group
}
\end{catsec}


\begin{dated}[Student Evaluations (Qualitative)]
    PHIL010 & The handouts were much more efficient than other styles of teaching I have experienced. You could follow along without any confusion of where we were, and you never were behind in taking notes. Also, the demonstrations and visual representation of the topics on the chalk board made it much easier to follow along instead of just listening to lecture. It was also very helpful that we covered each topic in the same manner. The order of definition, to clarification, to examples, to creating your own examples consistently helped me learn. - SU2015\\

    PHIL010 & Consistent teaching and class format, material was engaging and the course was the perfect balance of interesting material that the professor explained in a engaging and clear way, and an appropriate difficulty level to the point where the course was never too easy and boring and also never too stressful. – Fall 2020\\

    PHIL012 & Of all online courses I have taken at Penn State, professor Agler has been the most successful at teaching the course and providing ample resources to students, despite the difficult online platform. While the material is difficult to understand, Agler’s textbook breaks down the information in a very helpful way. His online discussion board is an invaluable resource to students that find themselves having questions on the material. Even more so, Agler’s response time (usually within a few hours) proves his dedication and true care for his students’ success. While sometimes difficult to explain, he takes the time to write back to every student’s question. He keeps his students alert of upcoming assignments (something that is easy to forget with an online course) by sending out regular emails. - SP2015\\
    PHIL012 & Normally with online courses, I am nervous about the amount of contact I will be able to have with the professor. Professor Agler extends himself to all students and makes sure they are able to contact him and get further help on the course material. He provides continuous updates and is proactive with his students. - SP2015\\
    PHIL012 & Dr. Agler was the most important aspect. I'm not really a math person and was a little bit apprehensive of a class that has many math elements to it, but Dr. Agler sets you up for success by giving you the tools you need to succeed: breakdowns of complex problems in class, detailed videos that he's made online, the textbook and packet which explain just about everything necessary, etc. Dr. Agler has set up the course in a way that if you want to succeed and you put the work in, you almost certainly will. You can tell he's very passionate about the subject by his willingness to help and go above and beyond for his students. He's also a genuinely good and funny guy, which helps him connect with the class and makes class entertaining - FA2021\\
    PHIL012 & Agler was very realistic throughout the course regarding pace and content he was comfortable altering the schedule and material to accommodate the needs of the class. This allowed for extensive review and increased comprehension, especially with the more difficult content in the course. He has a great sense of humor and connected with the students on a level that most college professors are unable to achieve. As a student who ordinarily struggles with math-related classes, he provided an environment that allowed me to excel past my expectations. The youtube videos were also extremely helpful when we were initially learning about truth trees. Overall, interesting (but challenging) class and awesome professor. - FA2015\\
    PHIL012 & The professor was very willing to help out students and re-word the explanations to better understand the concept. He would ALWAYS give real life examples to each new concept to help people understand at an easier level. He also found a tutor for students as well because this is known to be a relatively difficult course. He was an overall kind and understanding guy. - FA2015\\
    PHIL012 & As we near the end of the course, I wanted to thank you for all of your generosity throughout the semester. In my [...] years at Penn State and [...] years of
    school prior, I have never had an educator more willing to help students learn. Online classes tend to be more challenging, and having you as a professor made this course significantly less stressful. I was challenged throughout the entire course, but was helped in a way that I actually learned concepts I will take with me. As I study for the LSAT, many techniques of logic learned in this course are extraordinarily helpful. You consistently helped me do my best throughout the course and you never made me feel stupid or like I was bothering you, even though I emailed you multiple times each week. Thank you for being such an outstanding professor, and your eagerness to help students is refreshing. I truly appreciate all of the additional help and encouragement you provided throughout the semester. - SP2017\\
    PHIL012 & Professor Agler makes this class enjoyable. I dreaded taking this class but I can honestly say that I actually enjoyed it. Agler is a very personable guy and it makes a huge difference. Willing to help at anytime and clear up things that his students may be confused on. I would recommend Agler to anyone who has to take this course. – Fall 2021\\
    PHIL012 & David is enthusiastic about what and how he teaches material in class. He makes it interesting and easy to engage with. He provides good examples and can clearly break down complex topics so that we can more easily grasp the bigger picture. He also offers extra credit to encourage and emphasize the importance of practicing the material on our own. He is accommodating and understands that life can get in the way, and he is happy to reschedule exams or accept late work with little to no penalty. – Fall 2021\\
    PHIL012 & The homework that was due before each exam and the study guides we used before each exam helped me do well in the course. The professor was also very good at making sure you understood what was being and offered many opportunities to get additional help. The YouTube videos were also helpful. – Fall 2021\\
    PHIL012 & I love this course, and the professor. I consistently think about how well versed Dr. Agler is in these topics while learning about them in class. He truly cares about his students, and provides all the necessary tools in order to succeed in this course; for example practice exams, videos, definitions, textbooks, etc. Symbolic Logic is not easy to understand, however Dr. Agler's eloquence and care makes this course a piece of cake. He is intelligent, and is very well equipped in translating a (basically) foreign language \& all of it's intricacies within the span of 1 semester. He provides different perspectives for different students to ensure that everyone can understand the same topic in different ways. He is accommodating and understanding, responds to emails fast, and genuinely wants to work with students so that they can succeed. I truly cannot praise this professor enough. Honestly, it amazes me that a person can explain these topics so well in a logical order. I have absorbed the material of this course and have applied it to everyday use without thinking twice. If there is any way to promote him, please you should award it to him. - FA2023\\

    PHIL013 & The instructor’s enthusiasm and high energy, as well as the interesting material covered in the class made me look forward to coming to class. He did an excellent job explaining diffcult concepts through humorous examples. The instructor also did a great job challenging the students and encouraging them to think critically
    by posing very interesting questions. I also appreciated how eager the instructor was in hearing our feedback for the course and his willingness to implement the suggestions. - FA2014\\
    PHIL013 & The teacher used many different ways to get his point across, which helped understand a lot of lessons. He used videos, drawings on the blackboard, lecture, and group thinking to get ideas across. Talking among a group made me learn the most because it helped to hear other people’s interpretation of what was just being lectured. - FA2014\\

    PHIL123 & Via email: My name is [redacted], and I was a student in your Leap Media Ethics course. I am emailing you to say thank you for making the most out of the summer course! I learned so much through it and I felt more prepared in my journalism classes this semester. Whenever I had a question about journalism, I would use your old slides and the textbook of the course. Your class truly helped me with my transition into college life, especially the leap activities. From those activities, I was able to form a list of organizations I wanted to be a part of and internship opportunities. I am currently a news anchor for Penn State Network Television and a digital content intern for Penn State Athletics. I wouldn’t have been aware of all these opportunities without your help. Thank you so much for everything and I hope you had a great semester! – Spring 2021\\
    PHIL123& David is an amazing teacher who loved to help the class understand every single topic. He clearly cares about his students which really allowed us to care about him and the course in return. – Summer 2021\\
    PHIL123 &	The professor was wonderful. The learning environment was friendly, the course content was informational, and the communication was exceptional. Although I did not meet Mr. Agler in person, he seems like one of the kindest professors who cares about his students. It was obvious that he wanted us to be successful in his class-- he taught us so much, and he was always up for answering questions, whether it was through email or setting up a zoom. He seems to really love teaching, and his positive attitude is contagious! I'm glad I took his class, it was informational and enjoyable. I can't say enough positive things about him and his class! – Summer 2019\\
    PHIL297 & I’m going to use this space to express how much I liked David because there isn’t one. David was awesome. I really enjoyed his lectures, the way he explained things, and his passion for the subjects. Great work! - FA2016\\
    PHIL205 & Our professor really explained ideas thoroughly, used clear examples, and printed out note packets that helped us follow along. Overall, I learned a lot in this course and I really enjoyed it. \\
    PHIL205 & The packets that were handed out during class made it easy to follow lectures and understand the important points within the class. I also loved the way the class assignments were organized because it made me prepare for the essay I had to write. Also, the assignments made sure I understood the material we were learning.  – Spring 2019\\
    PHIL205 & Professor Agler made some conceptually challenging concepts very clear. He always had an example that made things make sense. He also has a very good sense of humor, and incorporates it as best he can with lectures. Also, I think he did a great job once we moved to remote learning; he clearly has some internet prowess. – Spring 2019\\
    PHIL205 & What helped me learn in this course was the thoroughness of Professor Agler's teaching. He made sure that everyone understood some difficult concepts without making the students incompetent. I had no prior knowledge to anything philosophy based and I feel that I have learned so much in this class. – Spring 2020\\
    PHIL401 & The lectures were thought out and organized, and the corresponding handouts made it much easier to navigate my own notes after class. I really appreciate the effort in organizing the material in these handouts and in the lectures themselves. The inclusion of some visual charts to communicate some relations between concepts was helpful. Also helpful was the instructor's activeness in sharing interesting and relevant materials besides the primary literature outside of class. – Fall 2018\\
\end{dated}
\end{document}